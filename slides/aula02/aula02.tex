\documentclass{beamer}
% Load Packages
\usepackage[utf8]{inputenc}
\usepackage{xcolor}
\usepackage{tikz}
\usetikzlibrary{positioning,calc}
\usepackage[absolute,overlay]{textpos}
\usepackage{graphicx}
\usepackage{hyperref}
\usepackage{amsmath}
\usepackage{listings}
\usepackage{fontawesome}
\usepackage[brazilian]{babel}
\usepackage{datetime}
\usepackage{subcaption} % usado para figuras lado a lado

% Define Commands
\newcommand*{\ClipSep}{0.06cm} %To adjust footer logo
\newcommand{\E}{\mathrm{e}\,} %\def\I{e} % used to defined e for exp(x), see later what it should be
\newcommand{\ud}{\mathrm{d}}
\lstset{numbers=left, numberstyle=\tiny, stepnumber=1,firstnumber=1,breaklines=true,
    numbersep=5pt,language=Python,
    stringstyle=\ttfamily,
    basicstyle=\footnotesize, 
    showstringspaces=false
}
 
\usetheme{oxonian}
\setbeamertemplate{itemize subitem}{$-$}
\setbeamertemplate{itemize subsubitem}[circle]
\setbeamercolor{block body alerted}{bg=alerted text.fg!10}
\setbeamercolor{block title alerted}{bg=alerted text.fg!20}
\setbeamercolor{block body}{bg=structure!10}
\setbeamercolor{block title}{bg=structure!20}
\setbeamercolor{block body example}{bg=green!10}
\setbeamercolor{block title example}{bg=green!20}
\setbeamertemplate{blocks}[rounded][shadow]
 
\title{Introdução a Orientação a Objetos - Parte II}
\subtitle{Programação Estruturada e Orientada a Objetos}
\titlegraphic{\includegraphics[width=2.5cm]{Theme/Logos/ifrn_cm_vertical.png}}
\author{Prof. Me. Leo Moreira Silva}

\date{\today}

\begin{document}

{\setbeamertemplate{footline}{} 
\frame{\titlepage}}

\section*{Agenda}\begin{frame}{Agenda}\tableofcontents\end{frame}

\section{Construtores}

\begin{frame}[fragile]{Construtores}
    \begin{itemize}
        \item Métodos importantes em classes
        \item São executados assim que o objeto é instanciado;
        \item Em Python, possui a seguinte estrutura:
    \end{itemize}
    \small
    \begin{semiverbatim}
        def __init__(self):
    \end{semiverbatim}
    \begin{itemize}
        \item Geralmente utilizado para a inialização de atributos.
    \end{itemize}
\end{frame}

\begin{frame}[fragile]{Construtores}
    \begin{itemize}
        \item Classe Carro \textbf{sem} construtor:
    \end{itemize}
    \small
    \begin{semiverbatim}
    class Carro:
        cor = 'sem cor'
        marca = 'sem marca'
        modelo = 'sem modelo'
        ano = 2010
        km_rodados = 0
    
        def detalhes(self):
            print('cor:', self.cor)
            print('marca:', self.marca)
            print('modelo:', self.modelo)
            print('ano:', self.ano)
            print('km rodados:', self.km_rodados)
    \end{semiverbatim}
\end{frame}

\begin{frame}[fragile]{Construtores}
    \scriptsize
    \begin{semiverbatim}
    >>>car_1 = Carro() #Instancia o objeto da classe Carro na variável 'car_1'
    >>>car_1.cor = 'Vermelho'
    >>>car_1.marca = 'Honda'
    >>>car_1.modelo = 'HR-V'
    >>>car_1.ano = 2016
    >>>car_1.detalhes() #Chama o método 'detalhes' implementado na classe Carro
    
    cor: Vermelho
    marca: Honda
    modelo: HR-V
    ano: 2016
    km_rodados: 0
    \end{semiverbatim}
\end{frame}

\begin{frame}[fragile]{Construtores}
    \begin{itemize}
        \item Classe Carro \textbf{com} construtor:
    \end{itemize}
    \small
    \begin{semiverbatim}
    class Carro:
        def __init__(self, cor, marca, modelo, ano, km_rodados):
            self.cor = cor
            self.marca = marca
            self.modelo = modelo
            self.ano = ano
            self.km_rodados = km_rodados
    
        def detalhes(self):
            print 'cor:', self.cor
            print 'marca:', self.marca
            print 'modelo:', self.modelo
            print 'ano:', self.ano
            print 'km rodados:', self.km_rodados
    \end{semiverbatim}
\end{frame}

\begin{frame}[fragile]{Construtores}
    \small
    \begin{semiverbatim}
    >>> from Carro_construtor import Carro
    >>> car = Carro('Azul', 'Honda', 'HR-V', 2016, 2000)
    >>> car.detalhes()
    
    cor: Azul
    marca: Honda
    modelo: HR-V
    ano: 2016
    km_rodados: 2000
    \end{semiverbatim}
\end{frame}

\section{Documentação}

\begin{frame}[fragile]{Documentação}
    \begin{itemize}
        \item Documentar Classes e Métodos
        \begin{itemize}
            \item Necessário para todos que irão utilizar o código
        \end{itemize}
        \item Função \texttt{help()}:
        \begin{itemize}
            \item Exibe documentação de um método/classe
        \end{itemize}
    \end{itemize}
    \small
    \begin{semiverbatim}
    >>> help(math.cos)
    Help on built-in function cos in module math:
    cos(...)
        cos(x)
        Return the cosine of x (measured in radians).
    \end{semiverbatim}
\end{frame}

\begin{frame}[fragile]{Documentação}
    \begin{itemize}
        \item Documentação em Python: \textbf{docstrings}
    \end{itemize}
    \small
    \begin{semiverbatim}
    class Carro:
        '''
        Classe que representa um carro. Cada carro possui:
            -cor
            -marca
            -modelo
            -ano
            -km_rodados
            -statusMotor
            -statusMovimento
        '''
    \end{semiverbatim}
\end{frame}

\begin{frame}[fragile]{Documentação}
    \small
    \begin{semiverbatim}
    def andar(self):
        '''Método que coloca o carro em movimento
        Verifica antes se o carro está ligado ou desligado'''
        if(self.statusMotor == True):
            if(self.statusMovimento == True):
                print 'O carro já está em movimento!'
            else:
                self.statusMovimento = True
                print 'Carro em movimento!'
        else:
            print 'Necessário ligar o motor!'
    \end{semiverbatim}
\end{frame}

\begin{frame}[fragile]{Documentação}
    \scriptsize
    \begin{semiverbatim}
    class Carro
        | Classe que representa um carro.
        | Cada carro possui:
        |   -cor
        |   -marca
        |   -modelo
        |   -ano
        |   -km_rodados
        |   -statusMotor
        |   -statusMovimento
        |
        | Methods defined here:
        |
        | andar(self)
        |   Método que coloca o carro em movimento
        |   Verifica antes se o carro está ligado ou desligado
    \end{semiverbatim}
\end{frame}

\section{Encapsulamento}

\begin{frame}{Encapsulamento}
    \begin{itemize}
        \item Encapsulamento de dados é a proteção dos atributos e métodos de uma Classe
        \item Seu objetivo é restringir o acesso direto à informação
        \item Existem dois tipos de atributos em OO - Python:
        \begin{itemize}
            \item Público
            \item Privado
        \end{itemize}
    \end{itemize}
\end{frame}

\begin{frame}{Encapsulamento}
    \begin{itemize}
        \item Atributos Públicos
        \begin{itemize}
            \item Podem ser acessados diretamente
            \item Não existe restrição quanto a escrita e leitura deles
        \end{itemize}
        \item Atributos privados
        \begin{itemize}
            \item São acessados via métodos
            \item Restrição de leitura e escrita aos dados de forma direta
        \end{itemize}
    \end{itemize}
\end{frame}

\begin{frame}[fragile]{Encapsulamento}
    \small
    \begin{semiverbatim}
    class Carro:
        #Atributo público
        cor = 'azul'

        #Atributo privado
        \_\_nomeProprietario = 'Leo Silva'
    \end{semiverbatim}
\end{frame}

\begin{frame}[fragile]{Encapsulamento}
    \begin{itemize}
        \item Se o atributo é privado, como acessar?
        \item Via \textbf{métodos}
    \end{itemize}
    \small
    \begin{semiverbatim}
    class Carro:
        #Atributo público
        cor = 'azul'
        #Atributo privado
        __nome_proprietario = 'Leo Silva'
        
        def get_nome_proprietario(self):
            return self.__nome_proprietario
        
        def set_nome_proprietario(self, novo_nome_proprietario):
            self.__nome_proprietario = novo_nome_proprietario
    \end{semiverbatim}
\end{frame}

\begin{frame}{Encapsulamento}
    \begin{itemize}
        \item Métodos GET/SET:
        \begin{itemize}
            \item Utilizados para acesso de leitura (GET) e escrita (SET) de atributos privados
        \end{itemize}
        \item Métodos definidos da seguinte forma:
        \begin{itemize}
            \item get\_nome\_do\_atributo(self)
            \item set\_nome\_do\_atributo(self, novo\_nome\_do\_atributo)
        \end{itemize}
    \end{itemize}
\end{frame}

\begin{frame}{Dúvidas?}
    \begin{figure}
        \includegraphics[scale=0.35]{Theme/Logos/duvidas_frequentes.png}
    \end{figure}
\end{frame}

\end{document}